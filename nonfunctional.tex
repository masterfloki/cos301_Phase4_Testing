\subsection{Maintainability:}

\subsection{Scalability:}

\subsection{Performance:}

\subsection{Reliability and Availability:}
Many functions in the implementation of the use-cases for Authentication B returns the correct result objects, but does not provide a synchronous way to access these returned objects. These functions reach a return statement before a database query is completed and therefore returns an undefined result, which is not at all reliable, since the function is useless if its result cannot be accessed.

\subsection{Security:}
In certain implementations of both  Group A and B, the system was made vulnerable, in that, checks for whether a user is an administrator were not done. This would allow any user, authorized or not, to access, remove or edit the authorization restrictions.

\subsection{Auditability:}
The authentication module from both group A and B do not meet the auditability requirement. No attempt was made to write to log files when any of the services were requested.

\subsection{Testability:}

\subsection{Usability:}

\subsection{Integrability:}
All functions/use-cases of Authentication B is exported into a NodeJS module which enhances integrability into other modules.

\subsection{Deployability:}