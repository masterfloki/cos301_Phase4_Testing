\subsection{Maintainability:}
A's system is set out in a way that it would be easy to maintain if you had an idea of the systematic outline of the code and if you had full understanding of the technologies used in the system. The documentation seemingly was more aimed at the team who implemented it instead of at future teams who might need to update it. Some comments are vague in the sense that a person who is trying to understand the code and does not yet have the bigger picture in mind, will not necessarily make sense of the code at first glance. Also not all the sections were documented properly.
B's system has more sufficient documentation, but it was not done throughout the whole system. It would be easy to understand the implementation where it was documented and the system as a whole makes sense, but where the system was not documented properly, if you do not understand the system as a whole, you will not understand the code at first glance. 

\subsection{Reliability and Availability:}
Many functions in the implementation of the use cases for Authentication B returns the correct result objects, but do not provide a synchronous way to access these returned objects. These functions reach a return statement before a database query is completed and therefore returns an undefined result, which is not at all reliable, since the function is useless if its result cannot be accessed. 
The implementations of the use-cases for Authentication A provides functionality to return results synchronously.
A single point of failure and availability violation for both modules is the connection with the database. In the highly unlikely event that this fails, results cannot be computed. Authentication B handles this relatively well for most use-case implementations by throwing exceptions to alert the caller that database connection caused the function to fail. There are however a few use-case implementations for which this was not provided. Authentication A provides no explicit means to detect a connection problem or to alert the caller if one occurs.

\subsection{Security:}
In certain implementations of both  Group A and B, the system was made vulnerable, in that, checks for whether a user is an administrator were not done. This would allow any user, authorized or not, to access, remove or edit the authorization restrictions.

\subsection{Auditability:}
The authentication module from both group A and B do not meet the auditability requirement. No attempt was made to write to log files when any of the services were requested.

\subsection{Testability:}
Authentication A includes all its testing implementation inside the app.js file and does not use unit.js for its unit tests.  
Authentication module for B has a test file but does not make any use of any unit test using unit.js. The test file contains heavily commented out code but when all the test code is uncommented out, majority of the code generates errors and some tests generate inconclusive results such as returning empty objects. With consideration of such factors, Authentication B module does not meet the requirements for testability and violates the quality testability requirement by not implementing the appropriate unit testing using unit.js.

\subsection{Integratability:}
All functions/use-case implementations of Authentication B is exported into a NodeJS module which enhances integrability into other modules.
The functions/use-case implementations of Authentication A is not exported into a module. This makes integrability tedious and difficult, since code will have to be copied to be used.