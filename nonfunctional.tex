\subsection{Maintainability:}
A's system is set out in a way that it would be easy to maintain if you have an idea of the systematic outlay of the code and if you have full understanding of the technologies used in the system. The documentation seemingly was more aimed at the team who implemented it instead of at future teams who might need to update it. Some comments are vague in the sense of a person who is trying to understand the code and does not yet have the bigger picture in mind, will not necessarily make sense of the code at first glance. Also not all the sections were documented probably.
B's system has more sufficient documentation, but it was not done through out the whole system. It would be easy to understand the implementation where it was documented and the system as a whole makes sense, but where the system was not documented properly, if you do not understand the system as a whole, you will not understand the code at first glance. 

\subsection{Scalability:}
Seemingly on average in both cases the functions used in both systems A and B, the complexity seems to be only log(n), hence it follows that these systems should scale well enough to accommodate the increase of users in this specific subsystem. 

\subsection{Performance:}

\subsection{Reliability and Availability:}
Many functions in the implementation of the use-cases for Authentication B returns the correct result objects, but does not provide a synchronous way to access these returned objects. These functions reach a return statement before a database query is completed and therefore returns an undefined result, which is not at all reliable, since the function is useless if its result cannot be accessed.

\subsection{Security:}
In certain implementations of both  Group A and B, the system was made vulnerable, in that, checks for whether a user is an administrator were not done. This would allow any user, authorized or not, to access, remove or edit the authorization restrictions.

\subsection{Auditability:}
The authentication module from both group A and B do not meet the auditability requirement. No attempt was made to write to log files when any of the services were requested.

\subsection{Testability:}
Authentication A includes all its testing implementation inside the app.js file and does not use unit.js to for its unit tests.  
Authentication module for B has a test file but does not make any use of any unit test using unit.js. The test file contains heavily commented out code but when all the test code is uncommented out, majority of the code generates errors and some tests generate inconclusive results such as returning empty objects. With consideration of such factors, Authentication B module does not meet the requirements for testability and violates the quality testability requirement by not implementing the appropriate unit testing using unit.js.

\subsection{Usability:}
The Authorization module meets the usability requirement because it consists
of methods that are easy to use in any part of the buzz system.
The "require" feature of nodeJS allows for effortless inclusion of the Authorization 
module, again this allows for easy integration with other modules.
The parameters of the methods in Authentication B are simple and straightforward and thus
quickly understood and implemented. The same applies to the methods in Authentication A, 
which only takes in RestrictionRequests as parameters, this prevents confusion and potential
errors due to mixed up parameters. 
The users of the buzz space don't come face to face with the backend of Authentication and as 
such are not considered in the usability of this module. However, the methods in this module
(Authentication A and B) do return clear and comprehensive results such as objects and exceptions. 
These results are in turn handled by other modules that interact with the users, so Authorization 
is indirectly quite usable to the extent that it can perform optimally through other modules.
The pre-conditions and post-conditions in both Authentication A and B ensure that the use of this 
module is done in a restricted manner so as to guide the user (admin) of this module, this again
makes it easy and simple to use.

\subsection{Integrability:}
All functions/use-cases of Authentication B is exported into a NodeJS module which enhances integrability into other modules.

\subsection{Deployability:}