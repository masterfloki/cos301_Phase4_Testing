\textbf{Description of use case:}\newline
Group A.
This use case is meant to facilitates editing of authorization restrictions of the database of A.
\medskip

\textbf{Violations:}
\begin{enumerate}

	\item Pre-conditions:
	\begin{itemize}
		\item There are no pre-conditions unmet
	\end{itemize}
	
	\item Post-conditions:
	\begin{itemize}
		\item There are no post conditions unmet
	\end{itemize}
		
\end{enumerate}

\textbf{Successes:}
\begin{itemize}

	
\item Pre-conditions:
	\begin{itemize}
		\item User is Administrator of space for which authorization restrictions are queried (i.e. he has permission/role)
			\begin{itemize}
					\item There is a check to see if this precondition is satisfied
			\end{itemize}
	\end{itemize}
		
	\item Post-conditions:
		\begin{itemize}
			\item The updates are persisted
		\end{itemize}
		
	\item Data structure:
		\begin{itemize}
			\item The data structures are correct in the database and all specified classes
		\end{itemize}
\end{itemize}

\noindent
\medskip

\textbf{Further information on Group A's updateAuthorizationRestrictions
 use case:}
\begin{itemize}
	\item The use case accepts a updateAuthorizationRestrictionRequest, which requires the instantiation of 2 different classes, the BuzzSpace class and a role class that the request object is composed of.

\item The implementation of this use case was done very simply, as it retrieves authorization restrictions, which in turn uses the service identifier,  by first checking if a user has the permission to update database and thereafter; the authorization restrictions are retrieved and returned a updateAuthorizationRestrictionResult as a JSON object.

\item The implementation does not check for the instance where the minimum status points are set to zero and it is in this case that the role name of a user can be used to determine the Service Restriction to see if he/she can update the Authorization restrictions.

\item If it was not possible to retrieve the authorization restrictions, as a result of a database error, then an error is simply logged onto the console screen, as no exceptions were required to have been thrown.

\item There are measures taken to ensure that the invariance constraint is met; therefore leaving the system in a consistent state.

\end{itemize}

\textbf{Description of use case:}\newline
Group B.
This use case is meant to facilitates editing of authorization restrictions of the database of B
\medskip

\textbf{Violations:}
\begin{enumerate}

	\item Pre-conditions:
	\begin{itemize}
		\item There are no pre-conditions unmet. 
	\end{itemize}
	
	\item Post-conditions:
	\begin{itemize}
		\item There are no post conditions unmet. New authorizations restrictions are persisted to database.
	\end{itemize}
		
\end{enumerate}

\textbf{Successes:}
\begin{itemize}

	
\item Pre-conditions:
	\begin{itemize}
		\item It is checked if user has appropriate permissions and role to update database. Otherwise user is notified that he doesn't.
			\begin{itemize}
					\item There is a check to see if this precondition is satisfied
			\end{itemize}
	\end{itemize}
		
	\item Post-conditions:
		\begin{itemize}
			\item The updates are persisted
		\end{itemize}
		
	\item Data structure:
		\begin{itemize}
			\item The data structures are not correct in the database and all specified classes are not there, but it still works like it should. It is done correctly how it's implemented. There is no updateAuthorzationRequest and updateAuthorizationResult classes. 
		\end{itemize}
\end{itemize}

\noindent
\medskip

\textbf{Further information on Group B's updateAuthorizationRestrictions
 use case:}
\begin{itemize}
	\item The use case does not use a updateAuthorizationRestrictionRequest, instead Role and , which requires the instantiation of 2 different classes, the BuzzSpace class and a role class that the request object is composed of.

\item The implementation of this use case was done very simply, as it retrieves authorization restrictions, which in turn uses the service identifier,  by first checking if a user has the permission to update database and the role and thereafter; the authorization restrictions are retrieved and then if succeeded sends as a JSON object through.

\item The implementation does not check for the instance where the minimum status points are set to zero and it is in this case that the role name of a user can be used to determine the Service Restriction to see if he/she can update the Authorization restrictions.

\item If it was not possible to retrieve the authorization restrictions, as a result of a database error, an exception was thrown to alert the user.

\item There are measures taken to ensure that the invariance constraint is met; therefore leaving the system in a consistent state.

\end{itemize}







