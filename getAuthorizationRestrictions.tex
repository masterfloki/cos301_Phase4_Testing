\textbf{Description of use case:}\newline
This use case is meant to be used by the front-end of the system in order to retrieve the authorization restrictions, so as to enable users to select an authorization restriction that they would like to update.
\medskip

\medskip

\noindent
The following test cases were tested on the getAuthorizationRestrictions use case:
\begin{enumerate}
	\item User's status points are greater than the service's minimum status points.
	\item User's status points are less than the service's minimum status points.
	\item User does not exist.
  	\item Service does not exist.
  	\item Buzz Space is not active.
  	\item Buzz Space does not exist.
  	\item User is Administrator for Buzz Space.
\end{enumerate}
\medskip

\noindent
\textbf{Group A's getAuthorizationRestrictions use case results:}\newline

\begin{itemize}
\item The following test cases passed: 1., 2., 3., and 4.
\item The following test cases failed: 5., 6. and 7.:
	\begin{itemize}
		\item No checks are made to ensure that a Buzz Space is active or even exists; thus authorization restrictions can be accessed outside of a Buzz Space.
		\item As no checks are done for if a user is an Administrator, any user can access, remove and edit the authorization restrictions in the database.
	\end{itemize}
	
\end{itemize}
\medskip

\noindent
\textbf{Violations:}
\begin{enumerate}
	\item Pre-conditions:	
	\begin{itemize}
		\item Space is active
		\begin{itemize}
				\item There is no explicit check as to whether this pre-condition is satisfied
		\end{itemize}
		\item User is Administrator of space for which authorization restrictions are queried
		\begin{itemize}
				\item There is no check to see if this pre-condition is satisfied
		\end{itemize}
	\end{itemize}
\end{enumerate}	
\medskip

\noindent
\textbf{Successes:}
\begin{enumerate}
\item Post-conditions:
	\begin{itemize}
		\item The use case returns a GetAuthorizationRestrictionResult as a JSON object containing the available Authorization Restrictions in the database.
	\end{itemize}

\item Data structure Requirements:
	\begin{itemize}
		\item The appropriate data structures are used for each field of a class
		\item The appropriate classes are created as specified by the class diagram in the requirements
	\end{itemize}

\end{enumerate}

\textbf{Further information on the use case:}
\noindent
\medskip

\begin{itemize}
	\item The function accepts a GetAuthorizationRestrictionRequest, which requires the instantiation of 4 different classes that the request object is composed of.

	\item The implementation of this use case was done very simply, as it retrieves authorization restrictions by first checking if a particular module exists in the database and thereafter; the authorization restrictions are retrieved and returned a GetAuthorizationRestrictionResult as a JSON object.

	\item The implementation does not check for the instance where the minimum status points are set to zero and it is in this case that the role name of a user should be used to determine the Service Restriction. 

	\item If it was not possible to retrieve the authorization restrictions, as a result of a database error, then an error is simply logged onto the console screen, as no exceptions were required to have been thrown.

	\item There were no measures taken to ensure that the invariance constraint is met; therefore leaving the system in an inconsistent state.
\end{itemize}


\medskip

\noindent
\textbf{Group B's getAuthorizationRestrictions use case results:}\newline

\begin{itemize}
\item The following test cases passed:  5 and 6.
	
\item The following test cases failed: 1, 2, 3, 4 and 7.
	\begin{itemize}
		\item No means is provided to even identify the user of this use-case. Status points of the user can therefore not be compared to the service's minimum status points.
		\item No validation is made to check that the current user actually exists
		\item No validation is made to ensure that the user is an administrator, since no means of even identifying the use-case user is provided.
	\end{itemize}

	\item An implementation summary of the use case is as follows: A restriction is found according to a Buzzspace id provided as an argument to the function. If the space does not exist or is not active a Buzzspace Error exception is thrown. If the space does exist and is active, the restriction objects is returned in the format of its database entry as a JSON. This is however only visible when printed to the console from within the implementation, due to the asynchronous nature of NodeJS.

\item If connection with the database cannot be established a Connection Error exception is thrown.

\end{itemize}

\textbf{Violations:}
\begin{enumerate}
	\item Pre-conditions:	
	\begin{itemize}
		\item User is Administrator of space for which authorization restrictions are queried
		\begin{itemize}
				\item No validation is done to ensure the user querying the restrictions is an administrator
		\end{itemize}
	\end{itemize}
	
	\item Post-conditions:
	\begin{itemize}
		\item  A GetAuthorizationRestrictionResult object should be returned by the function. 
		\begin{itemize}
				\item Due to the asynchronous nature of Nodejs, the result of the function is undefined when called although the result is later available. The function returns before it has computed a result and no means to return the result synchronously is provided. Therefore this post-condition is violated.
		\end{itemize}
	\end{itemize}
	
	\item Data structure requirement:
	\begin{itemize}
		\item  A GetAuthorizationRestrictionResult object should be returned by the function. 
		\begin{itemize}
				\item The argument passed to GetAuthorizationRestriction in the implementation of Authorization B is a string containing a Buzzspace ID instead of a GetAuthorizationRestrictionRequest object. Therefore this is a data structure requirement violation.
		\end{itemize}
	\end{itemize}
		
\end{enumerate}

\textbf{Successes:}
\begin{enumerate}
	\item Pre-conditions:	
	\begin{itemize}
		\item Space is active
		\begin{itemize}
				\item The function only selects documents from the database where the delete parameter is false to indicate the space is active.
		\end{itemize}
	\end{itemize}	
\end{enumerate}


\noindent
\medskip




