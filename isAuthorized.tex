\textbf{Description of use case:}\newline
This use case is meant to customize the user interface by querying the services the user may access.
\medskip

\noindent
The following test cases were tested on the isAuthorized use case:
\begin{enumerate}
	\item User's status points are greater than the service's minimum status points.
	\item User's status points are less than the service's minimum status points.
	\item User does not exist.
  	\item Service does not exit.
  	\item Buzz Space is not active.
  	\item Buzz Space does not exist.
\end{enumerate}
\medskip
\noindent
\textbf{Group A's isAuthorized use case results:}\newline
The function takes in a isAuthorizedRequest object as a argument and returns a isAuthorizedResult if all pre- and post- conditions were met.\newline
Passed the following test cases: 1, 2 and 4.\newline
Failed the following test cases: 3, 5 and 6.\newline
When testing isAuthorized with a inactive space a isAuthorizedResult with a boolean value true was returned, this is incorrect as it should have thrown a SpaceNotActive exception. The pre-condition, space is active, was violated.
\medskip

\noindent
\textbf{Group B's isAuthorized use case results:}\newline
The function did not follow the master specification and could not be tested as it was very difficult to determine the correct argument(s) to pass through when calling isAuthorized. 