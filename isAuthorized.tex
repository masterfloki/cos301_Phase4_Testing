\textbf{Description of use case:}\newline
This use case is meant to customize the user interface by querying the services the user may access. The service contract for the isAuthorized use case specifies that isAuthorized is called with a isAuthorizedRequest and returns a isAuthorizedResult which contains a boolean value that indicates whether the user is authorized to use a particular service. isAuthorized has only one pre-condition and that is the space, the user is on, be active(open). A SpaceNotActiveException must be thrown if the pre-condition was violated.
\medskip

\noindent
The following test cases were tested on the isAuthorized use case:
\begin{enumerate}
	\item User's status points are greater than the service's minimum status points.
	\item User's status points are less than the service's minimum status points.
	\item User does not exist.
  	\item Service does not exit.
  	\item Buzz Space is not active.
  	\item Buzz Space does not exist.
\end{enumerate}
\medskip
\noindent
\textbf{Group A's isAuthorized use case results:}\newline
Passed the following test cases: 1, 2 and 4.\newline
Failed the following test cases: 3, 5 and 6.\newline
When testing isAuthorized with an inactive space a isAuthorizedResult with a boolean value true was returned, this is incorrect as it should have thrown a SpaceNotActive exception. The pre-condition, space is active, was violated.
\medskip

\noindent
\textbf{Group B's isAuthorized use case results:}\newline
Could not test group B's isAuthorized use case as the master specification was not followed. Multiple arguments were required to call isAuthorized. Determining the correct required arguments to pass through could not be done because the master specification was not followed.