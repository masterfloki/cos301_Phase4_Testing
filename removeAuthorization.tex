\textbf{Description of use case:}\newline
This is a simple use case which removes an authorization restriction for a user role from a buzz space.


\medskip

\noindent
Test cases for removeAuthorization:
\begin{enumerate}
	\item User can access Buzz Space.
	\item User enters appropriate parameters and recieves approriate response
	\item User with a certain role can remove authorization.

\end{enumerate}

\noindent
\textbf{Authentication B results}\newline
Success/Violation of the contract requirements(pre-conditions):
\begin{itemize}
	\item removeAuthorization is able to establish a connection to the database to obtain the necessary details for a particular user.
\end{itemize}

\noindent
Success/Violation of the contract requirements(post-conditions):
\begin{itemize}
	\item The function can successfully apply the remove restriction by evaluating the role of a user. If the user does not have the certain role required they are simple declined access to the functionality.
\end{itemize} 

\noindent
Violation of the contract requirements(post-conditions):
\begin{itemize}
	\item When a user is declined access from removing a restriction, a message is return indicating that the user does not have the rights to perform such an operation. This meets the contract requirements.
\end{itemize}

\noindent
Data Structure Success/Violation:
\begin{itemize}
	\item The input parameter in the contract requirements requires the object removeAuthorizationRestrictionRequest be passed to the function which encapsulates the userId. The implemented function violates the requirements by passing a string that contains the userId.
	\item The function is supposed to return removeAuthorizationRestrictionResult object after the removal of the specified restriction but the implmemented code only returns an empty object which voilates the requirements.
\end{itemize} 











