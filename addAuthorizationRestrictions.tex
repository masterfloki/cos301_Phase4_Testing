
\textbf{Description of use case:}\newline
\par{The domain model for buzzAuthorization specifies that a restriction is made for a particular user role on a specific buzz space.

This is not the case in Authentication A and B's addAuthorizationRestrictions:}

\begin{itemize}
\item Authentication A: A restriction is not added, but updated according to the role and buzz space. 
\item Authentication B: A Restriction is instead added to a specific service on a buzzSpace with a role and minimum status points.
\end{itemize}
\medskip

\textbf{Functional testing/assesment Success or Violations:}
\begin{enumerate}
	\item Pre-conditions:
	
	\begin{itemize}
		\item User adding the restriction has admin rights:
		\begin{itemize}
				\item Authentication A: This condition was met, because it tests for the admin rights in the updateAuthorizationRestriction which was called in this use case.
				\item Authentication B: This was not met in
this use case, because the service doesn't test for the condition at all.
		\end{itemize}
		\item The restriction does not yet exist:
		\begin{itemize}
				\item Authentication A: This condition is not met according to the master spec, because if the restriction exists it updates the restriction. 
				\item Authentication B: This pre-condition is met in the service;
it tests if the restriction exists and if so an appropriate exception is thrown,
if not then the restriction specified is added/persisted to the database after a few conditions are met.
		\end{itemize}
	\end{itemize}

	\item Post-conditions:
	\begin{itemize}
		\item The restriction persisted:
			\begin{itemize}
			\item Authentication A: The condition isn't met, because it throws an exception when the restriction doesn't exist. This is because it calls the update method which requires the restrictions to already exist.
			\item Authentication B: This condition is met when all the pre-conditions in the use case are met (those specified in the master specification and those that are not), then the restriction is persisted to the database.				
			\end{itemize}
			
	\end{itemize}
		
\end{enumerate}

\noindent
\medskip

\textbf{Further information on addAuthorizationRestrictions use case:}
\begin{itemize}
	\item	Pre-conditions found in the usecase of team-B that wasn't specified:
	\textnormal{Although not specified, they are needed for the usecase to work.}
		\begin{itemize}
		\item Specified buzz space should exist.  
		\item Specified role should be valid and exist.
		\item The service should exist.
		\item Minimum status points should be valid, being >= 0.
		\item Usecase should be able to connect to database. 
		\end{itemize}
	\item The use case of team-A takes in 1 argument:
		\begin{itemize}
		\item The argument is an object containing all the fields that need to be added to the restriction.
		
		\end{itemize}
	\item The use case of team-B takes in 4 arguments:
		\begin{itemize}
		\item The buzz space ID.
		\item The service's ID.
		\item A role.
		\item A minimum status points		
		\end{itemize}
	  

\end{itemize}





