\textbf{Description of use case:}\newline
This use case is meant to be used by the front-end of the system in order to retrieve the authorization restrictions, so as to enable users to select an authorization restriction that they would like to update.
\medskip

\textbf{Violations:}
\begin{enumerate}
	\item Pre-conditions:
	
	\begin{itemize}
		\item Space is active
		\begin{itemize}
				\item There is no explicit check as to whether this precondition is satisfied
		\end{itemize}
		\item User is Administrator of space for which authorization restrictions are queried
		\begin{itemize}
				\item There is no check to see if this precondition is satisfied
		\end{itemize}
	\end{itemize}
	
	\item Post-conditions:
	\begin{itemize}
		\item There were no post-conditions specified for this use case
	\end{itemize}
		
\end{enumerate}

\textbf{Successes:}
\begin{itemize}
	\item The appropriate data structures are used for each field of a class
	\item The appropriate classes are created as specified by the class diagram in the requirements
\end{itemize}


\noindent
\medskip

\textbf{Further information on Group A's getAuthorizationRestrictions
 use case:}
\begin{itemize}
	\item The use case accepts a GetAuthorizationRestrictionRequest, which requires the instantiation of 4 different classes that the request object is composed of.

\item The implementation of this use case was done very simply, as it retrieves authorization restrictions by first checking if a particular module exists in the database and thereafter; the authorization restrictions are retrieved and returned a GetAuthorizationRestrictionResult as a JSON object.

\item The implementation does not check for the instance where the minimum status points are set to zero and it is in this case that the role name of a user can be used to determine the Service Restriction. 

\item If it was not possible to retrieve the authorization restrictions, as a result of a database error, then an error is simply logged onto the console screen, as no exceptions were required to have been thrown.

\item There are no measures taken to ensure that the invariance constraint is met; therefore leaving the system in an inconsistent state.

\end{itemize}





